\documentclass[10pt]{article}
\usepackage[a4paper,margin=1cm,landscape]{geometry}
\usepackage[utf8]{inputenc}
\usepackage[french]{babel}
\usepackage{libertine}
\usepackage[T1]{fontenc}
\usepackage{xcolor}
\usepackage{tikz}
\usepackage{graphicx}

\newcommand*\circled[1]{
  \tikz[baseline=(char.base)]{
    \node[shape=circle,draw,fill=white,inner sep=2pt ] (char) {#1};}}

\title{Modèle <<Squad\footnote{Le terme <<squad>> (traduit ici par <<équipe>>) est le mot utilisé par Spotify pour une équipe de développement petite, cross-fonctionnelle et auto-organisée.}
  Health Check>> - Version française {\small basé sur la version 1 de septembre 2014}}
\date{}

\begin{document}

\maketitle

\begin{tikzpicture}[remember picture, overlay]
  \node [shift={(-100 mm,-70 mm)}] at (current page.north east) {
    \includegraphics[width=6cm]{_output/hand}
  };
  \node [shift={(-120 mm,-85 mm)}] at (current page.north east) {
    \Huge\circled{1}
  };
\end{tikzpicture}

\begin{tikzpicture}[remember picture, overlay]
  \node [shift={(-40 mm,-80 mm)}] at (current page.north east) {
    \includegraphics[width=5cm]{rules/includes/equipe}
  };
  \node [shift={(-58 mm,-105 mm)}] at (current page.north east) {
    \Huge\circled{2}
  };
\end{tikzpicture}

\begin{tikzpicture}[remember picture, overlay]
  \node [shift={(-54 mm,-145 mm)}] at (current page.north east) {
    \includegraphics[width=8cm]{rules/includes/evolution}
  };
  \node [shift={(-68 mm,-165 mm)}] at (current page.north east) {
    \Huge\circled{3}
  };
\end{tikzpicture}

\begin{minipage}{0.6\linewidth}
\section*{De quoi s’agit-il ?}
Un atelier et une technique de visualisation aidant les équipes à s’améliorer.

\section*{Audience ?}
\begin{itemize}
  \item L’équipe elle-même
  \item Les personnes apportant leur support à l’équipe
  \item (managers, coachs, etc.)
\end{itemize}


\section*{Comment utiliser ce modèle}
\begin{itemize}
  \item Rassemblez tous les membres de l’équipe dans la même salle
  \item \circled{1} : Discutez sur les cartes de questions. Chacune d’entre elle est un indicateur de bonne santé, accompagné d’un exemple de très bonne performance et d’un exemple particulièrement inefficace.
  \item \circled{2} : Demander à l’équipe comment elle se positionne sur chacun de ces indicateurs, en utilisant une méthode favorisant les décisions de groupe (par exemple : avec les cartes de vote).
  \item \circled{3} : Discutez les tendances d’évolution de ces indicateurs (la situation s’améliore-t-elle ? Est- elle stable ou se dégrade-t-elle ?)
  \item Matérialisez visuellement les résultats de ces discussions.
  \item Utilisez des données quantitatives (estimation, mesures, extrapolation...) pour aider l’équipe à s’améliorer.
\end{itemize}


\section*{Idées de mises en œuvre}
  Les cartes sont uniquement un point de départ pour initialiser des conversations productives. L’équipe doit se sentir libre d’ajouter/ôter/modifier toute question afin de correspondre à ce qu’elle considère comme important pour elle.

  Assurez-vous que cet outil soit utilisé en support de l’équipe dans son amélioration et surtout pas pour l’évaluer!
\end{minipage} \hfill

\begin{tikzpicture}[remember picture, overlay]
  \node [shift={(-56 mm,-19 cm)}] at (current page.north east) {
    \noindent\fcolorbox{lightgray}{lightgray}{
      \begin{minipage}{9 cm}
        \section*{\normalsize Crédits}
        \scriptsize
        \begin{itemize}
          \item Modèle Health check : Henrik Kniberg \& Kristian Lindwall, https://labs.spotify.com/2014/09/16/squad-health-check-model/
          \item Traduction française : Thomas Clavier, Séverin Legras, Hervé Taboucou, avec l’aide des membres d'Ajiro.fr
          \item Design des cartes : Olivier Albiez, Nicolas Leconte
          \item Images: http://emojione.com/
        \end{itemize}
        Distribuez, modifiez, réutilisez ces contenus sous licence CC~BY-SA~3.0~FR
      \end{minipage}
    }
  };
\end{tikzpicture}


\end{document}
